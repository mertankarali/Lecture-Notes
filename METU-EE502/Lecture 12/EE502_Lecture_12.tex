% 


\documentclass[twoside]{article}
\setlength{\oddsidemargin}{0.25 in}
\setlength{\evensidemargin}{-0.25 in}
\setlength{\topmargin}{-0.6 in}
\setlength{\textwidth}{6.5 in}
\setlength{\textheight}{8.5 in}
\setlength{\headsep}{0.75 in}
\setlength{\parindent}{0 in}
\setlength{\parskip}{0.1 in}

%
% ADD PACKAGES here:
%

\usepackage{amsmath,amsfonts,graphicx,arydshln}


\newcounter{lecnum}
\renewcommand{\thepage}{\thelecnum-\arabic{page}}
\renewcommand{\thesection}{\thelecnum.\arabic{section}}
\renewcommand{\theequation}{\thelecnum.\arabic{equation}}
\renewcommand{\thefigure}{\thelecnum.\arabic{figure}}
\renewcommand{\thetable}{\thelecnum.\arabic{table}}

%
% The following macro is used to generate the header.
%
\newcommand{\lecture}[4]{
   \pagestyle{myheadings}
   \thispagestyle{plain}
   \newpage
   \setcounter{lecnum}{#1}
   \setcounter{page}{1}
   \noindent
   \begin{center}
   \framebox{
      \vbox{\vspace{2mm}
    \hbox to 6.28in { {\bf EE502 - Linear Systems Theory II
	\hfill Spring 2023} }
       \vspace{4mm}
       \hbox to 6.28in { {\Large \hfill Lecture #1 \hfill} }
       \vspace{2mm}
       \hbox to 6.28in { {\it Lecturer: #2 \hfill } }
      \vspace{2mm}}
   }
   \end{center}
   \markboth{Lecture #1}{Lecture #1}

   \vspace*{4mm}
}

\renewcommand{\cite}[1]{[#1]}
\def\beginrefs{\begin{list}%
        {[\arabic{equation}]}{\usecounter{equation}
         \setlength{\leftmargin}{2.0truecm}\setlength{\labelsep}{0.4truecm}%
         \setlength{\labelwidth}{1.6truecm}}}
\def\endrefs{\end{list}}
\def\bibentry#1{\item[\hbox{[#1]}]}


\newcommand{\fig}[3]{
			\vspace{#2}
			\begin{center}
			Figure \thelecnum.#1:~#3
			\end{center}
	}

% Use these for theorems, lemmas, proofs, etc.
\newtheorem{theorem}{Theorem}[lecnum]
\newtheorem{lemma}[theorem]{Lemma}
\newtheorem{proposition}[theorem]{Proposition}
\newtheorem{claim}[theorem]{Claim}
\newtheorem{corollary}[theorem]{Corollary}
\newtheorem{definition}[theorem]{Definition}
\newenvironment{proof}{{\bf Proof:}}{\hfill\rule{2mm}{2mm}}
\newtheorem{exmp}[theorem]{Ex}

% **** IF YOU WANT TO DEFINE ADDITIONAL MACROS FOR YOURSELF, PUT THEM HERE:

\begin{document}

% Lecture Details
\lecture{12}{Asst. Prof. M. Mert Ankarali}


%%%%%%%%%%%%%%%%%%%%%%%%%%

\section{The Kalman Decomposition}

In reachability and observability lectures, we derived two types of standards forms, 
specifically for unreachable systems and unobservable systems (separately). Now our goal is
to propose a general standard form for a unreachable and unobservable system, based on
the Kalman decomposition. The process is exactly same for bot DT and CT systems, thus 
we will present the decomposition for only CT systems. Let
%
\begin{align*}
  \dot{x} &= A x + B u \ , \ y &= C x + D u \ \& \   x \in \mathbb{R}^n
\end{align*}
%
Let's assume that system is neither reachable, nor observable and
%
\begin{align*}
  \mathrm{rank}[ \mathbf{R} ] = r < n \ , \ \mathrm{range}[ \mathbf{R} ] = \mathcal{R}
  \\
  \mathrm{dim}[ \mathcal{N} ( \mathbf{O} ) ] = \bar{o} > 0 \ , \ \bar{\mathcal{O}} = \mathcal{N} ( \mathbf{O} )
\end{align*}
%
Let's consider the following similarity transformation
%
\begin{align*}
  \hat{A} = T^{-1} A T \ , \ \hat{B} = T^{-1} B \ , \ \hat{C} = C T \ \& \ D = D
\end{align*}
%
Let  
%
\begin{align*}
  T = \left[ \begin{array}{c|c|c|c} T_{r\bar{o}} & T_{ro} & T_{\bar{r} \bar{o}} & T_{\bar{r} o} \end{array} \right]
\end{align*}
%
Let's define sub-matrices as follows:
%
\begin{enumerate}
\item Let $\mathcal{R}\bar{\mathcal{O}} = \mathcal{R} \cap \bar{\mathcal{O}}$, i.e. 
$x \in \mathcal{R}\bar{\mathcal{O}} \, \Rightarrow \, x \in \mathcal{R} \ \& \ x \in \bar{\mathcal{O}}$. 
Columns of $T_{r\bar{o}} $ forms a basis for $\mathcal{R}\bar{\mathcal{O}}$.
\end{enumerate}
%

% **** This ENDS THE EXAMPLES. DON'T DELETE THE FOLLOWING LINE:
\end{document}
