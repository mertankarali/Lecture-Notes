\documentclass[twoside]{article}
\setlength{\oddsidemargin}{0.25 in}
\setlength{\evensidemargin}{-0.25 in}
\setlength{\topmargin}{-0.6 in}
\setlength{\textwidth}{6.5 in}
\setlength{\textheight}{8.5 in}
\setlength{\headsep}{0.75 in}
\setlength{\parindent}{0 in}
\setlength{\parskip}{0.1 in}

%
% ADD PACKAGES here:
%

\usepackage{amsmath,amsfonts,graphicx,mathtools}

%
\newcounter{lecnum}
\renewcommand{\thepage}{\thelecnum-\arabic{page}}
\renewcommand{\thesection}{\thelecnum.\arabic{section}}
\renewcommand{\theequation}{\thelecnum.\arabic{equation}}
\renewcommand{\thefigure}{\thelecnum.\arabic{figure}}
\renewcommand{\thetable}{\thelecnum.\arabic{table}}

%
% The following macro is used to generate the header.
%
\newcommand{\lecture}[4]{
   \pagestyle{myheadings}
   \thispagestyle{plain}
   \newpage
   \setcounter{lecnum}{#1}
   \setcounter{page}{1}
   \noindent
   \begin{center}
   \framebox{
      \vbox{\vspace{2mm}
    \hbox to 6.28in { {\bf EE502 - Linear Systems Theory II
	\hfill Spring 2032} }
       \vspace{4mm}
       \hbox to 6.28in { {\Large \hfill Lecture #1 \hfill} }
       \vspace{2mm}
       \hbox to 6.28in { {\it Lecturer: #2 \hfill } }
      \vspace{2mm}}
   }
   \end{center}
   \markboth{Lecture #1}{Lecture #1}

   \vspace*{4mm}
}


\renewcommand{\cite}[1]{[#1]}
\def\beginrefs{\begin{list}%
        {[\arabic{equation}]}{\usecounter{equation}
         \setlength{\leftmargin}{2.0truecm}\setlength{\labelsep}{0.4truecm}%
         \setlength{\labelwidth}{1.6truecm}}}
\def\endrefs{\end{list}}
\def\bibentry#1{\item[\hbox{[#1]}]}

%Use this command for a figure; it puts a figure in wherever you want it.
%usage: \fig{NUMBER}{SPACE-IN-INCHES}{CAPTION}
\newcommand{\fig}[3]{
			\vspace{#2}
			\begin{center}
			Figure \thelecnum.#1:~#3
			\end{center}
	}
% Use these for theorems, lemmas, proofs, etc.
\newtheorem{theorem}{Theorem}[lecnum]
\newtheorem{lemma}[theorem]{Lemma}
\newtheorem{proposition}[theorem]{Proposition}
\newtheorem{claim}[theorem]{Claim}
\newtheorem{corollary}[theorem]{Corollary}
\newtheorem{definition}[theorem]{Definition}
\newenvironment{proof}{{\bf Proof:}}{\hfill\rule{2mm}{2mm}}
\newtheorem{exmp}[theorem]{Ex}

% **** IF YOU WANT TO DEFINE ADDITIONAL MACROS FOR YOURSELF, PUT THEM HERE:

\begin{document}

% Lecture Details
\lecture{7}{Assoc. Prof. M. Mert Ankarali}



\section{Discrete-Time Linear Time Varying State Space Models} 

State-space representation of a (causal \& finite dimensional) LTV DT system is given by
%
\begin{align*}
  \mathrm{Let} \ x[k] &\in \mathbb{R}^n \ , \ y[k] \in \mathbb{R}^m \ ,\  u[k] \in
  \mathbb{R}^r , \\
  x[k+1] &= A[k] x[k] + B[k] u[k] , \\
  y[k] &= C[l] x[k] + D u[k] , \\
  \mathrm{where} \ A[k] &\in \mathbb{R}^{n \times n} \ , \ 
    B[k] \in \mathbb{R}^{n \times r} \ ,\  C[k] \in \mathbb{R}^{m \times n} \ , \ D[d] \in \mathbb{R}^{m \times r}
\end{align*}
%
Let's first assume that $u[k] = 0$, and find un-driven 
response.
%
\begin{align*}
  x[k+1] &= A[k] x[k] \\ y[k] &= C[k] x[k]
\end{align*}
%
Unlike LTV-CT systems we easily can compute the response iteratively
%
\begin{align*}
   x[0] &= I x[0] \quad , \quad y[0] = C[0] x[0]
   \\
  x[1] &= A[0] x[0] \quad , \quad y[1] = C[1] x[1]
  \\
  x[2] &= A[0] x[1] = A[1] A[0] x[0] \quad , \quad y[2] = C[2] x[2]
  \\
  x[3] &= A[2] x[2] = A[3] A[1] A[0] x[0] \quad , \quad y[3] = C[3] x[3]
  \\
  \vdots
  \\
  x[k] &= A[k-1] x[k-1] = A[k-1] A[k-2] \cdots A[1] A[0] x[0]  \quad , \quad y[k] = \quad , \quad y[k] = C[k] x[k]
  \\
  x[k] &= \prod_{i=0}^{k-1} A[k-1-i] 
\end{align*}
%
Motivated by the LTI case, we define the \textbf{state transition matrix}, which relates the state of the
undriven system at time $k$ to the state at an earlier time $m$
%
\begin{align*}
  x[k] &= \Phi[k,m] x[m] \ , \  k \geq m \ , \  \mathrm{where}
  \\
  \Phi[k,m] &= \left\lbrace \begin{array}{l} \prod_{i=m}^{k-1} A[k-1-i] \ , \ k > m 
  \\ I \quad \quad \quad \quad  \quad \quad  \quad \ \ , \ k = m 
  \end{array} \right.
\end{align*}
%
Note that the state transition matrix satisfies the following important properties of the undriven system at time $k$ to the state at an earlier time $m$
%
\begin{align*}
  \Phi[k,k] &= I
  \\
  x[k] &= \Phi[k,0] x[0]
  \\
  \Phi[k+1,m] &= A[k] \Phi[k,m] 
\end{align*}
%
as you can see, the state-transition matrix satisfies the discrete dynamical state equations. 
Now let's consider input-only state response (i.e. $x[0] = 0$).
%
\begin{align*}
  x[k+1] &= A[k] x[k] + B[k] u[k] 
  \\
  \\
  x[1] &= B[0] u[0] = \Phi[1,1] B[0] u[0]
  \\
  \\
  x[2] &= A[1] x[1] + B[1] u[1] = A[1] B[0] u[0] + B[1] u[1]  = \Phi[2,1] B[0] u[0] + \Phi[2,2] B[1] u[1] 
  \\
  \\
  x[3] &= A[2] x[2] + B[2] u[2] = A[2] A[1] B[0] u[0] + A[2] B[1] u[1] + B[2] u[2]
  \\
  \\
  &= \Phi[3,1] B[0] u[0] + \Phi[3,2] B[1] u[1] + \Phi[3,3] B[2] u[2]
  \\
  \\
  x[4] &= A[3] x[3] + B[3] u[3] = A[3] A[2] A[1] B[0] u[0] + A[3] A[2] B[1] u[1] + A[3] B[2] u[2] + B[3] u[3]
  \\
  \\
  &= \Phi[4,1] B[0] u[0] + \Phi[4,2]  B[1] u[1] + \Phi[4,3] B[2] u[2] + \Phi[4,4] B[3] u[3]
  \\
  \vdots
 \\
  x[k] &= \Phi[k,1] B[0] u[0] + \Phi[k,2] B[1] u[1] + \cdots + \Phi[k,k-1] B[k-2] u[k-2] + \Phi[k,k] B[k-1] u[k-1]
         \\
         &= \left[ \begin{array}{c|c|c|c|c} \Phi[k,1] B[0] & \Phi[k,2] B[1]  & \cdots & \Phi[k,k-1] B[k-2]  & \Phi[k,k] B[k-1] \end{array} \right]
         \left[ \begin{array}{c}
                  u[0] \\ u[1] \\ \vdots \\ u[k-2] \\ u[k-1]
         \end{array} \right]
         \\
         &= \sum\limits_{j = 0}^{k-1} \Phi[k,j+1] B[j] u[j]
         \\
         &= \Gamma[k,0] \mathcal{U}[k,0]
\end{align*}
%
where
%
\begin{align*}
    \Gamma[k,0] &= \left[ \begin{array}{c|c|c|c|c} \Phi[k,1] B[0] & \Phi[k,2] B[1]  & \cdots & \Phi[k,k-1] B[k-2]  & \Phi[k,k] B[k-1] \end{array} \right]
\\
    \mathcal{U}[k,0] &=  \left[ \begin{array}{c} u[0] \\ u[1] \\ \vdots \\ u[k-2] \\ u[k-1]  \end{array} \right]
\end{align*}
%

\begin{exmp}
Let's consider following SISO LTP system
 \begin{align*}
  x[k+1] &= A[k] x[k] + B[k] u[k] , \ , y[k] = C[k] x[k]  \ , \   \mathrm{where} \\
  A[k] &= A[k+4]  \ , \  B[k] = B[k+4]  \ , \ C[k] = C[k+4]
 \end{align*}
\end{exmp}
%
Convert this SISO LTP system into a MISO LTI system

\textbf{Solution:} Let's derive $x[4]$ in terms of $x[0] , u[0] , u[1], u[2] , u[3]$

\begin{align*}
 x[4] &= A[3] A[2] A[1] A[0] x[0] +  A[3] A[2] A[1] B[0] u[0] + A[3] A[2] B[1] u[1] + A[3] B[2] u[2] + B[3] u[3]
 \\
 \\
  &= \begin{bmatrix} A[3] A[2] A[1] A[0] \end{bmatrix} x[0] + \begin{bmatrix} A[3] A[2] A[1] B[0] & A[3] A[2] B[1] & A[3] B[2] &  B[3] \end{bmatrix} 
 \begin{bmatrix} u[0] \\ u[1] \\ u[2] \\ u[3] \end{bmatrix} 
\end{align*}
%
Now let's find $x[8]$ in terms of $x[4] , u[4] , u[5], u[6] , u[7]$
%
\begin{align*}
 x[8] &= \begin{bmatrix} A[3] A[2] A[1] A[0] \end{bmatrix} x[4] + \begin{bmatrix} A[3] A[2] A[1] B[0] & A[3] A[2] B[1] & A[3] B[2] &  B[3] \end{bmatrix} 
 \begin{bmatrix} u[4] \\ u[5] \\ u[6] \\ u[7] \end{bmatrix} 
\end{align*}
%
Let $\mathcal{X}[m] = x[4k]$,  $\mathcal{U}[m] = \begin{bmatrix} u[4k] & u[4k + 1] & u[4k + 2] & u[4k + 3] \end{bmatrix}^T$, and $\mathcal{Y}[m] = y[4k]$, then we have 
%
\begin{align*}
 \mathcal{X}[m+1] &= \begin{bmatrix} A[3] A[2] A[1] A[0] \end{bmatrix} \mathcal{X}[m] + \begin{bmatrix} A[3] A[2] A[1] B[0] & A[3] A[2] B[1] & A[3] B[2] &  B[3] \end{bmatrix} 
\mathcal{U}[m] 
\\
 &= \mathcal{A}  \mathcal{X}[m] + \mathcal{B} \mathcal{U}[m] 
\\ 
\mathcal{Y}[m] &= \left[ C[0] \right] \mathcal{X}[m]
\\
&= \mathcal{C} \mathcal{X}[m]
\end{align*}
%
which is a multi-input single-output LTI state-space representation. Note that in this representation we technically perform periodic sampling and analyze the mapping between two sampling instants. However, in this representation, we technically lose some information, e.g. measurements/outputs between to sampling instants, that is, $y[4k+1] \ , \ y[4k+2] \, \ y[4k+3] \ k > 0$.

\begin{exmp}
	Obtain a new state-space representation (still LTI) such that it also includes these measurements, i.e. $y[4k+1] \ , \ y[4k+2] \, \ y[4k+3] \ k > 0$.
\end{exmp}

\newpage

\section{Continuous-Time Linear Time Varying State Space Models} 

State-space representation of a (causal \& finite dimensional) LTV CT system is given by
%
\begin{align*}
  \mathrm{Let} \ x(t) &\in \mathbb{R}^n \ , \ y(t) \in \mathbb{R}^q \ ,\  u(t) \in \mathbb{R}^r , \\
  \dot{x}(t) &= A(t) x(t) + B(t) u(t) , \\
  y(t) &= C(t) x(t) + D(t) u(t) , 
\end{align*}
%
If we assume that the mapping, $t \mapsto A(t)$ is sufficiently well behaved (if $\forall (i,j) \ , \ a_{ij}(t)$ is piecewise continuous, with a fnite number of discontinuities in any nite interval) there exist a unique solution for the differential equation. In such a case we can describe the solution of the system of equations via a matrix function, $\Phi(t , \tau)$, that the following 
two fundamental properties
%
\begin{align*}
	\frac{d}{dt} \Phi(t , \tau) &= A(t) \Phi(t , \tau) ,
	\\
	\Phi(\tau , \tau) &= I
\end{align*}
%
First let's show that $x(t) = \Phi(t , t_0) x_0$ for the solution of zero-input response with $x(t_0) = x_0$
%
\begin{align*}
	x(t_0) = \Phi(t_0 , t_0) x_0 = I x_0 = x_0 \ \checkmark
	\\
	\frac{d}{dt} x(t) = \frac{d}{dt} \left[ \Phi(t , t_0) x_0 \right] = A(t) \Phi(t , t_0 ) x_0 = A(t) x(t) , \ \checkmark
\end{align*}
%
Now let's show that $x(t) = \int\limits_{t_0}^{t} \Phi(t , \tau) B(\tau) u(\tau) d\tau$ for the solution of zero-state response 
%
\begin{align*}
	\frac{d}{dt} x(t) &= \frac{d}{dt} \left[ \int\limits_{t_0}^{t} \Phi(t , \tau) B(\tau) u(\tau) d\tau \right]
	\\
	&= \left[ \int\limits_{t_0}^{t} \frac{\partial}{\partial t} \Phi(t , \tau) B(\tau) u(\tau) d\tau \right] + \left[ \Phi(t , \tau) B(\tau) u(\tau) \right]_{\tau = t}
	\\
	&= \left[ \int\limits_{t_0}^{t} A(t) \Phi(t , \tau) B(\tau) u(\tau) d\tau \right] + \Phi(t , t) B(t) u(t) 
	\\
	&= A(t) \left[ \int\limits_{t_0}^{t} \Phi(t , \tau) B(\tau) u(\tau) d\tau \right] + I B(t) u(t) 
	\\
	\frac{d}{dt} x(t) &= A(t) x(t) + B(t) u(t) \ \checkmark
\end{align*}
%
It is relatively easy to show that 
%
\begin{align*}
	\Phi(t_2 , t_0) = \Phi(t_2 , t_1) \Phi(t_1 , t_0) \ , \ t_2 \geq t_1 \geq t_0
\end{align*}

\subsection{Solution of Scalar LTV Models}

The state-evolutiontion equation of a scalar LTV CT system is given by
%
\begin{align*}
  \dot{x}(t) &= a(t) x(t) + b(t) u(t) , \ x(t) \in \mathbb{R} , \ y(t) \in \mathbb{R} \ ,\  u(t) \in \mathbb{R} , 
\end{align*}
%
Let us try to find $x(t)$ and $\Phi(t,\tau)$ for a zero-input response.
%
\begin{align*}
  &\dot{x}(t) = a(t) x(t) \ \rightarrow \ \frac{1}{x} dx = a(t) dt \ \rightarrow \int\limits_{x(t_0)}^{x(t)} \frac{1}{x} dx = \int\limits_{t_0}^{t} a(\gamma) d\gamma \ \rightarrow \
  \left[ \mathrm{ln} x \right]_{x_0}^{x(t)} = \int\limits_{t_0}^{t} a(\gamma) d\gamma \ \rightarrow \   \mathrm{ln} (x / x_0 ) = \int\limits_{t_0}^{t} a(\gamma) d\gamma 
  \\
&x(t) = e^{ \int\limits_{t_0}^{t} a(\gamma) d\gamma  }  x_0 \ \rightarrow \ \Phi(t,\tau) = e^{ \int\limits_{\tau}^{t} a(\gamma) d\gamma  } 
\end{align*}
%
The generaleral solution then can be expressed as
%
\begin{align*}
x(t) &= \Phi(t , t_0) x_0 +  \int\limits_{t_0}^{t} \Phi(t , \tau) B(\tau) u(\tau) d\tau
\\
&= e^{ \int\limits_{t_0}^{t} a(\gamma) d\gamma  }  x_0 +  \int\limits_{t_0}^{t} \begin{pmatrix} e^{ \int\limits_{\tau}^{t} a(\gamma) d\gamma  } \end{pmatrix} B(\tau) u(\tau) d\tau
\end{align*}

\begin{exmp}
	Let
	\begin{align*}	
		\frac{d}{dt} x(t) = \begin{bmatrix} - t & 0 \\ 0 & \cos t \end{bmatrix} x(t) + \begin{bmatrix} 1 \\ 1  \end{bmatrix} u(t)
	\end{align*}	
\end{exmp}
%
First find the state-transition matrix

\textbf{Solution:} Even though the system is not a scalar LTV system since it is in diagonal form, we can use the same derivation on the scalar case
%
\begin{align*}	
	\Phi(t , t_0) = \begin{bmatrix} e^{ \int\limits_{t_0}^{t} -\gamma d\gamma  } & 0 \\ 0 & e^{ \int\limits_{t_0}^{t} \cos(\gamma) d\gamma  } \end{bmatrix} 
	= \begin{bmatrix} e^{ \frac{t_0^2 - t^2}{2} } & 0 \\ 0 & e^{ \sin(t) - \sin(t_0)  } \end{bmatrix} 
\end{align*}	
%
Now let $x(0) = 0$ and $u(t) = 1$, find $x(t)$
%
\begin{align*}
x(t) &= \int\limits_{0}^{t} \Phi(t , \tau) B(\tau) u(\tau) d\tau = \int\limits_{0}^{t} \begin{bmatrix} e^{ \frac{\tau^2 - t^2}{2} } & 0 \\ 0 & e^{ \sin(t) - \sin(\tau)  } \end{bmatrix}  \begin{bmatrix} 1 \\ 1  \end{bmatrix} d\tau
= \begin{bmatrix} \int\limits_{0}^{t} e^{ \frac{\tau^2 - t^2}{2} } d\tau \\ \int\limits_{0}^{t} e^{ \sin(t) - \sin(\tau)  } d\tau \end{bmatrix}   
\\
&= \begin{bmatrix} e^{ \frac{- t^2}{2} } \int\limits_{0}^{t} e^{ \frac{\tau^2}{2} } d\tau \\ e^{ \sin(t) } \int\limits_{0}^{t} e^{ - \sin(\tau)  } d\tau \end{bmatrix}   
\end{align*}

\subsection{Solution of Vector LTV Models}

\textbf{Question:} For the vector LTV systems, can we express the zero input solution and the state-transition matrix as
%
\begin{align*}
x(t) &\overset{?}{=}  e^{ \int\limits_{t_0}^{t} A(\gamma) d\gamma  }  x_0 \ , \ \Phi(t,\tau) \overset{?}{=}  e^{ \int\limits_{\tau}^{t} A(\gamma) d\gamma  } 
\end{align*}
%
Let $\Psi(t,0) = e^{ \int\limits_{0}^{t} A(\gamma) d\gamma  } $ and expand the matrix exponential 
%
\begin{align*}
\Psi(t,0) &= e^{ \int\limits_{0}^{t} A(\gamma) d\gamma }  = I + \begin{pmatrix} \int\limits_{0}^{t} A(\gamma) d\gamma \end{pmatrix} + \frac{1}{2 !} \begin{pmatrix} \int\limits_{0}^{t} A(\gamma) d\gamma \end{pmatrix}^2  
+ \frac{1}{3 !} \begin{pmatrix} \int\limits_{0}^{t} A(\gamma) d\gamma \end{pmatrix}^3  + \cdots 
\\ 
&= \sum\limits_{i=0}^{\infty} \frac{1}{i !} \begin{pmatrix} \int\limits_{0}^{t} A(\gamma) d\gamma \end{pmatrix}^i  
\end{align*}
%
Let's focus on the derivates of the individual elements in the expansion of $\Psi(t,0)$
%
\begin{align*}
\frac{d}{dt} I &= 0 \ , \ \checkmark
\\
\frac{d}{dt}  \begin{pmatrix} \int\limits_{0}^{t} A(\gamma) d\gamma \end{pmatrix} &= A(t) \ , \ \checkmark
\\
\frac{d}{dt} \frac{1}{2}  \begin{pmatrix} \int\limits_{0}^{t} A(\gamma) d\gamma \end{pmatrix}^2 &= \frac{1}{2} \frac{d}{dt} \left[  \begin{pmatrix} \int\limits_{0}^{t} A(\gamma) d\gamma \end{pmatrix}  \begin{pmatrix} \int\limits_{0}^{t} A(\gamma) d\gamma \end{pmatrix} \right] \\
&= \frac{1}{2} \left[ A(t) \begin{pmatrix} \int\limits_{0}^{t} A(\gamma) d\gamma \end{pmatrix} \right] 
+  \frac{1}{2} \left[ \begin{pmatrix} \int\limits_{0}^{t} A(\gamma) d\gamma \end{pmatrix} A(t) \right] 
\end{align*}
%
If $\begin{pmatrix} \int\limits_{0}^{t} A(\gamma) d\gamma \end{pmatrix}$ and $A(t)$ does not commute, we can go further from here. However, if $\begin{pmatrix} \int\limits_{0}^{t} A(\gamma) d\gamma \end{pmatrix}$ and $A(t)$ commutes then 
%
\begin{align*}
\frac{d}{dt} \frac{1}{2} \begin{pmatrix} \int\limits_{0}^{t} A(\gamma) d\gamma \end{pmatrix}^2 &=  A(t) \begin{pmatrix} \int\limits_{0}^{t} A(\gamma) d\gamma \end{pmatrix} 
= \begin{pmatrix} \int\limits_{0}^{t} A(\gamma) d\gamma \end{pmatrix} A(t) 
\\
\frac{d}{dt} \frac{1}{2 !} \begin{pmatrix} \int\limits_{0}^{t} A(\gamma) d\gamma \end{pmatrix}^3 &= \frac{1}{2 !}  A(t) \begin{pmatrix} \int\limits_{0}^{t} A(\gamma) d\gamma \end{pmatrix}^2
= \frac{1}{2 !} \begin{pmatrix} \int\limits_{0}^{t} A(\gamma) d\gamma \end{pmatrix}^2 A(t) 
\\
&\vdots
\end{align*}
%
Thus similar to constant $A$ case if $\begin{pmatrix} \int\limits_{0}^{t} A(\gamma) d\gamma \end{pmatrix}$ and $A(t)$ commutes, we can express the derivative of the matrix exponential as
%
\begin{align*}
\frac{d}{dt} \Psi(t,t_0) &= \frac{d}{dt} e^{ \int\limits_{t_0}^{t} A(\gamma) d\gamma }  = A(t) e^{ \int\limits_{t_0}^{t} A(\gamma) d\gamma } = e^{ \int\limits_{t_0}^{t} A(\gamma) d\gamma } A(t)
\end{align*}
% 
Now let us check if $\Psi(t,t_0) = \Phi(t,t_0)$, i.e. control weather $e^{ \int\limits_{t_0}^{t} A(\gamma) d\gamma }$ is a state transition matrix of the given LTV system
%
\begin{align*}
\Psi(t_0,t_0) = e^{ \int\limits_{0}^{0} A(\gamma) d\gamma } = I \ , \ \checkmark
\\
\frac{d}{dt} \Psi(t,\tau) &= A(t) \Psi(t,\tau) \ , \ \checkmark
\end{align*}
%
Thus, $\Phi(t,t_0)  = e^{ \int\limits_{t_0}^{t} A(\gamma) d\gamma }$ is the state transition matrix of the system under the condition that
$\begin{pmatrix} \int\limits_{0}^{t} A(\gamma) d\gamma \end{pmatrix}$ and $A(t)$ commute.

Two special cases for which $\begin{pmatrix} \int\limits_{0}^{t} A(\gamma) d\gamma \end{pmatrix}$ and $A(t)$ commute
%
\begin{itemize}
	\item $A(t)$ is a diagonal matrix $\forall t \in \mathbb{R}$
	\item $A(t) = A_c f(t)$ where $A_c \in \mathbb{R}^{n \times n}$ is a constant matrix and $f(t) \in \mathbb{R}$ is a scalar function.
\end{itemize}

\subsubsection{General Theory for CT LTV Systems}

Previously we showed that if $\begin{pmatrix} \int\limits_{0}^{t} A(\gamma) d\gamma \end{pmatrix}$ and $A(t)$ commutes we can write an ``analytical'' expression (for which we generally still adopt numerical integration techniques) for the state transition matrix, $\Phi(t,\tau) = e^{ \int\limits_{t_0}^{\tau} A(\gamma) d\gamma }$
For a general linear time-varying system, there is no analytical expression that expresses $\Phi(t,\tau)$ analytically as a function of $A(t)$. Instead, we are essentially limited to numerical solutions, how ever we still need a theoretical framework. 

\textbf{Definition:} Consider $\dot{x} = A(t) x$, where $A(t)$ is a sufficiently well behaved matrix of functions of $t$. Then for every initial state, 
$x(t_0)_i \ , \ i = 1, \ 2, \ \cdots n $, there exists a unique solution $x(t)_i \ , \ i = 1, \ 2, \ \cdots n $. Let's define a matrix 
%
\begin{align*}
X(t) \vcentcolon= \left[ \begin{array}{c|c|c|c} x(t)_1 & x(t)_2 & \cdots & x(t)_n \end{array} \right] \ , \ X(t) \in \mathbb{C}^{n}
\end{align*}
%
The matrix $X(t)$ satisfies the matrix differential equation, $\dot{X}(t) = A(t) X(t)$. If $X(t_0)$ is full rank (and hence invertible), then $X(t)$ is called ``a fundamental matrix''. Note that $X(t)$ is not unique and there exists infinitely many fundamental matrices. 

\vspace{6pt}

\textbf{Theorem:} A fundamental matrix $X(t)$ is invertible $\forall \ t$

\textbf{Proof:} Let's try proof by contradiction. 

Suppose that $\exists \ \bar{t}$ such that $\det( X(\bar{t}) ) = 0$, then we know that 
%
\begin{align*}
	\mathrm{dim} \mathcal{N}\left( X(\bar{t}) \right) > 0 \ \rightarrow \ \exists \ \bar{v}\neq0 \ s.t. \ X(\bar{t}) \bar{v} = 0
\end{align*}
%
now let $\zeta(t) = X(\bar{t}) \bar{v}$ and take the derivate of $\zeta(t)$
%
\begin{align*}
	\dot{\zeta}(t) &= \dot{X}(\bar{t}) \bar{v} = A(t) \left[ X(\bar{t}) \bar{v} \right] = A(t) \zeta(t)
	\\
	\dot{\zeta}(t) &= A(t) \zeta(t)
\end{align*}
%
Since we assumed that $X(\bar{t}) \bar{v} = 0$
%
\begin{align*}
	\zeta(\bar{t}) &= 0 \ \rightarrow \ \zeta(t) = 0 \ \mathrm{for} \ t < \bar{t} \ \mathrm{(CT-Linear-System)}
	\\
	\zeta(t_0) &= 0 \ \rightarrow \ X(t_0) \bar{v} = 0 \ \rightarrow \ \mathrm{det}[X(t_0)]=0 \ \rightarrow \ \mathrm{contrdiction}
\end{align*}
%
since, if $\mathrm{det}[X(t_0)]=0$ then $X(t)$ is not a fundamental matrix. 

\vspace{6pt}

\textbf{Theorem:} Let $X(t)$ be any fundamental matrix of $\dot{x} = A(t) x(t)$, then $\Phi(t,t_0) = X(t) X(t_0)^{-1}$

\textbf{Proof:} Let's show that $X(t) X(t_0)^{-1}$ is the state-transition matrix
%
\begin{align*}
\Phi(t_0,t_0) &= \left[ X(t) X(t_0)^{-1} \right]_{t=t_0} = X(t_0) X(t_0)^{-1}= I \ , \ \checkmark
	\\
	\\
	\frac{d}{dt} \Phi(t , \tau) &= \frac{d}{dt} \left[ X(t) X(\tau)^{-1} \right] = \dot{X}(t) X(\tau)^{-1}  = A(t) X(t) X(\tau)^{-1}
	\\
	\frac{d}{dt} \Phi(t , \tau) &= A(t) \Phi(t , \tau) \ , \ \checkmark
\end{align*}

\begin{exmp}
Let 
\begin{align*}
	\dot{x}(t) = \begin{bmatrix} 0 & t & 0 \\ 0 & 0 & t \\ 0 & 0 & 0 \end{bmatrix} x(t)
\end{align*}
\end{exmp}
%
Compute $\Phi(t,0)$ via two different methods

\textbf{Solution:} Let's first find state-transition matrix via finding a fundamental matrix. Let $X(0) = I$ 
%
\begin{align*}
	X(0) = \begin{bmatrix} 1 & 0 & 0 \\ 0 & 1 & 0 \\ 0 & 0 & 1 \end{bmatrix} 
	= \begin{bmatrix} e_1 & e_2 & e_3  \end{bmatrix} 
\end{align*}
% 
Let's start with $x(0) = e_1$. Let's write individual differential equations in the reverse order of the state numbering 
%
\begin{align*}
	\dot{x}_3 &= 0 \ , \ x_3(0) = 0 \ \rightarrow \ x_3(t) = 0
	\\
	\dot{x}_2 &= t x_3 = 0\ , \ x_2(0) = 0 \ \rightarrow \ x_2(t) = 0
	\\
	\dot{x}_1 &= t x_2 = 0 \ , \ x_1(0) = 1 \ \rightarrow \ x_1(t) = 1
	\\
	x(t)_1 &= \begin{bmatrix} 1 \\ 0 \\ 0 \end{bmatrix}
\end{align*}
% 
Now apply the same process for $x(0) = e_2$.
%
\begin{align*}
	\dot{x}_3 &= 0 \ , \ x_3(0) = 0 \ \rightarrow \ x_3(t) = 0
	\\
	\dot{x}_2 &= t x_3 = 0 \ , \ x_2(0) = 1 \ \rightarrow \ x_2(t) = 1
	\\
	\dot{x}_1 &= t x_2 = t \ , \ x_1(0) = 0 \ \rightarrow \ x_1(t) = \frac{t^2}{2}
	\\
	x(t)_2 &= \begin{bmatrix} \frac{t^2}{2} \\ 1 \\ 0 \end{bmatrix}
\end{align*}
%
Finally apply the same process for $x(0) = e_3$.
%
\begin{align*}
	\dot{x}_3 &= 0 \ , \ x_3(0) = 1 \ \rightarrow \ x_3(t) = 1
	\\
	\dot{x}_2 &= t x_3 = t \ , \ x_2(0) = 0 \ \rightarrow \ x_2(t) =  \frac{t^2}{2}
	\\
	\dot{x}_1 &= t x_2 = \frac{t^3}{2} \ , \ x_1(0) = 0 \ \rightarrow \ x_1(t) = \frac{t^4}{8}
	\\
	x(t)_3 &= \begin{bmatrix} \frac{t^4}{8} \\ \frac{t^2}{2} \\ 1 \end{bmatrix}
\end{align*}
%
We can then form the fundamental and state-transition matrices 
%
\begin{align*}
	X(t) &= \begin{bmatrix} x(t)_1 & x(t)_2 & x(t)_3 \end{bmatrix}  = \begin{bmatrix} 1 & \frac{t^2}{2} & \frac{t^4}{8} \\ 0 & 1 & \frac{t^2}{2} \\ 0 & 0 & 1 \end{bmatrix} 
	\\
	\Phi(t,0) &= X(t)X(0)^{-1} = \begin{bmatrix} 1 & \frac{t^2}{2} & \frac{t^4}{8} \\ 0 & 1 & \frac{t^2}{2} \\ 0 & 0 & 1 \end{bmatrix} 
\end{align*}
% 
Now, let's find the state-transition matrix using the matrix exponential. Since $\begin{pmatrix} \int\limits_{0}^{t} A(\gamma) d\gamma \end{pmatrix}$ and $A(t)$ commutes, we can express the state-transition matrix as
%
\begin{align*}
	\Phi(t,0) &= e^{ \int\limits_{0}^{t} A(\gamma) d\gamma } 
	= \mathrm{exp} \begin{pmatrix} \int\limits_{0}^{t} \begin{bmatrix} 0 & \gamma & 0 \\ 0 & 0 & \gamma \\ 0 & 0 & 0 \end{bmatrix} d\gamma \end{pmatrix}	
	= \mathrm{exp} \begin{pmatrix} \begin{bmatrix} 0 & \frac{t^2}{2} & 0 \\ 0 & 0 & \frac{t^2}{2} \\ 0 & 0 & 0 \end{bmatrix} \end{pmatrix}	
	\\
	\Phi(t,0) &= \begin{bmatrix} 1 & \frac{t^2}{2} & \frac{t^4}{8} \\ 0 & 1 & \frac{t^2}{2} \\ 0 & 0 & 1 \end{bmatrix} 
\end{align*}

\begin{exmp}
Compute the unit step response for the same system given that
\begin{align*}
	x(0) = 0 \ , \ B(t) = \begin{bmatrix} 0 \\ 0 \\ 1 \end{bmatrix} 
	\ , \
	C(t) = \begin{bmatrix} 1 & 0 & 0 \end{bmatrix} 
\end{align*}
\end{exmp}
%
\textbf{Solution:} First let's find $\Phi(t , \tau)$
%
\begin{align*}
	\Phi(t , \tau) &= X(t) X(\tau)^{-1} = \begin{bmatrix} 1 & \frac{t^2}{2} & \frac{t^4}{8} \\ 0 & 1 & \frac{t^2}{2} \\ 0 & 0 & 1 \end{bmatrix} 
	\begin{bmatrix} 1 & \frac{\tau^2}{2} & \frac{\tau^4}{8} \\ 0 & 1 & \frac{\tau^2}{2} \\ 0 & 0 & 1 \end{bmatrix}^{-1}
	= \begin{bmatrix} 1 & \frac{t^2}{2} & \frac{t^4}{8} \\ 0 & 1 & \frac{t^2}{2} \\ 0 & 0 & 1 \end{bmatrix} 
	\begin{bmatrix} 1 & -\frac{\tau^2}{2} & \frac{\tau^4}{8}  \\ 0 & 1 & -\frac{\tau^2}{2} \\ 0 & 0 & 1 \end{bmatrix}
	\\
\end{align*}
%
Now we can find the solution.
%
\begin{align*}
	y(t) &= C(t) \int\limits_{t_0}^{t} \Phi(t , \tau) B(\tau) u(\tau) d\tau 
	\\
	&= \begin{bmatrix} 1 & 0 & 0 \end{bmatrix}  \int\limits_{0}^{t} \begin{bmatrix} 1 & \frac{t^2}{2} & \frac{t^4}{8} \\ 0 & 1 & \frac{t^2}{2} \\ 0 & 0 & 1 \end{bmatrix} 
	\begin{bmatrix} 1 & -\frac{\tau^2}{2} & \frac{\tau^4}{8}  \\ 0 & 1 & -\frac{\tau^2}{2} \\ 0 & 0 & 1 \end{bmatrix} \begin{bmatrix} 0 \\ 0 \\ 1 \end{bmatrix}  d\tau 
	\\
	&= \begin{bmatrix} 1 & \frac{t^2}{2} & \frac{t^4}{8} \end{bmatrix}  \int\limits_{t_0}^{t} \begin{bmatrix} \frac{\tau^4}{8} \\ -\frac{\tau^2}{2} \\ 1 \end{bmatrix} d\tau 
	= \begin{bmatrix} 1 & \frac{t^2}{2} & \frac{t^4}{8} \end{bmatrix} \begin{bmatrix} \frac{t^5}{40} \\ -\frac{t^3}{6} \\ t \end{bmatrix} 
	\\
	&= \frac{t^5}{40} - \frac{t^5}{12} + \frac{t^5}{8} = \frac{t^5}{15}
\end{align*}
%


% **** This ENDS THE EXAMPLES. DON'T DELETE THE FOLLOWING LINE:
\end{document}
