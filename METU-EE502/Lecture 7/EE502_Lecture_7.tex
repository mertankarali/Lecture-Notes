

\documentclass[twoside]{article}
\setlength{\oddsidemargin}{0.25 in}
\setlength{\evensidemargin}{-0.25 in}
\setlength{\topmargin}{-0.6 in}
\setlength{\textwidth}{6.5 in}
\setlength{\textheight}{8.5 in}
\setlength{\headsep}{0.75 in}
\setlength{\parindent}{0 in}
\setlength{\parskip}{0.1 in}

%
% ADD PACKAGES here:
%

\usepackage{amsmath,amsfonts,graphicx}

%
\newcounter{lecnum}
\renewcommand{\thepage}{\thelecnum-\arabic{page}}
\renewcommand{\thesection}{\thelecnum.\arabic{section}}
\renewcommand{\theequation}{\thelecnum.\arabic{equation}}
\renewcommand{\thefigure}{\thelecnum.\arabic{figure}}
\renewcommand{\thetable}{\thelecnum.\arabic{table}}

%
% The following macro is used to generate the header.
%
\newcommand{\lecture}[4]{
   \pagestyle{myheadings}
   \thispagestyle{plain}
   \newpage
   \setcounter{lecnum}{#1}
   \setcounter{page}{1}
   \noindent
   \begin{center}
   \framebox{
      \vbox{\vspace{2mm}
    \hbox to 6.28in { {\bf EE502 - Linear Systems Theory II
	\hfill Spring 2032} }
       \vspace{4mm}
       \hbox to 6.28in { {\Large \hfill Lecture #1 \hfill} }
       \vspace{2mm}
       \hbox to 6.28in { {\it Lecturer: #2 \hfill } }
      \vspace{2mm}}
   }
   \end{center}
   \markboth{Lecture #1}{Lecture #1}

   \vspace*{4mm}
}


\renewcommand{\cite}[1]{[#1]}
\def\beginrefs{\begin{list}%
        {[\arabic{equation}]}{\usecounter{equation}
         \setlength{\leftmargin}{2.0truecm}\setlength{\labelsep}{0.4truecm}%
         \setlength{\labelwidth}{1.6truecm}}}
\def\endrefs{\end{list}}
\def\bibentry#1{\item[\hbox{[#1]}]}

%Use this command for a figure; it puts a figure in wherever you want it.
%usage: \fig{NUMBER}{SPACE-IN-INCHES}{CAPTION}
\newcommand{\fig}[3]{
			\vspace{#2}
			\begin{center}
			Figure \thelecnum.#1:~#3
			\end{center}
	}
% Use these for theorems, lemmas, proofs, etc.
\newtheorem{theorem}{Theorem}[lecnum]
\newtheorem{lemma}[theorem]{Lemma}
\newtheorem{proposition}[theorem]{Proposition}
\newtheorem{claim}[theorem]{Claim}
\newtheorem{corollary}[theorem]{Corollary}
\newtheorem{definition}[theorem]{Definition}
\newenvironment{proof}{{\bf Proof:}}{\hfill\rule{2mm}{2mm}}
\newtheorem{exmp}[theorem]{Ex}

% **** IF YOU WANT TO DEFINE ADDITIONAL MACROS FOR YOURSELF, PUT THEM HERE:

\begin{document}

% Lecture Details
\lecture{7}{Assoc. Prof. M. Mert Ankarali}



\section{Discrete-Time Linear Time Varying State Space Models} 

State-space representation of a (causal \& finite dimensional) LTV DT system is given by
%
\begin{align*}
  \mathrm{Let} \ x[k] &\in \mathbb{R}^n \ , \ y[k] \in \mathbb{R}^m \ ,\  u[k] \in
  \mathbb{R}^r , \\
  x[k+1] &= A[k] x[k] + B[k] u[k] , \\
  y[k] &= C[l] x[k] + D u[k] , \\
  \mathrm{where} \ A[k] &\in \mathbb{R}^{n \times n} \ , \ 
    B[k] \in \mathbb{R}^{n \times r} \ ,\  C[k] \in \mathbb{R}^{m \times n} \ , \ D[d] \in \mathbb{R}^{m \times r}
\end{align*}
%
Let's first assume that $u[k] = 0$, and find un-driven 
response.
%
\begin{align*}
  x[k+1] &= A[k] x[k] \\ y[k] &= C[k] x[k]
\end{align*}
%
Unlike LTV-CT systems we easily can compute the response iteratively
%
\begin{align*}
   x[0] &= I x[0] \quad , \quad y[0] = C[0] x[0]
   \\
  x[1] &= A[0] x[0] \quad , \quad y[1] = C[1] x[1]
  \\
  x[2] &= A[0] x[1] = A[1] A[0] x[0] \quad , \quad y[2] = C[2] x[2]
  \\
  x[3] &= A[2] x[2] = A[3] A[1] A[0] x[0] \quad , \quad y[3] = C[3] x[3]
  \\
  \vdots
  \\
  x[k] &= A[k-1] x[k-1] = A[k-1] A[k-2] \cdots A[1] A[0] x[0]  \quad , \quad y[k] = \quad , \quad y[k] = C[k] x[k]
  \\
  x[k] &= \prod_{i=0}^{k-1} A[k-1-i] 
\end{align*}
%
Motivated by the LTI case, we define the \textbf{state transition matrix}, which relates the state of the
undriven system at time $k$ to the state at an earlier time $m$
%
\begin{align*}
  x[k] &= \Phi[k,m] x[m] \ , \  k \geq m \ , \  \mathrm{where}
  \\
  \Phi[k,m] &= \left\lbrace \begin{array}{l} \prod_{i=0}^{k-1} A[k-1-i] \ , \ k > m 
  \\ I \quad \quad \quad \quad  \quad \quad  \quad \ \ , \ k = m 
  \end{array} \right.
\end{align*}
%
Note that state-transition matrix satisfies following important properties
undriven system at time $k$ to the state at an earlier time $m$
%
\begin{align*}
  \Phi[k,k] &= I
  \\
  x[k] &= \Phi[k,0] x[0]
  \\
  \Phi[k+1,m] &= A[k] \Phi[k,m] 
\end{align*}
%
as you can see, the state-transition matrix satisfies the discrete dynamical state equations. 
Now let's consider input-only state response (i.e. $x[0] = 0$).
%
\begin{align*}
  x[k+1] &= A[k] x[k] + B[k] u[k] 
  \\
  \\
  x[1] &= B[0] u[0] = \Phi[1,1] B[0] u[0]
  \\
  \\
  x[2] &= A[1] x[1] + B[1] u[1] = A[1] B[0] u[0] + B[1] u[1]  = \Phi[2,1] B[0] u[0] + \Phi[2,2] B[1] u[1] 
  \\
  \\
  x[3] &= A[2] x[2] + B[2] u[2] = A[2] A[1] B[0] u[0] + A[2] B[1] u[1] + B[2] u[2]
  \\
  \\
  &= \Phi[3,1] B[0] u[0] + \Phi[3,2] B[1] u[1] + \Phi[3,3] B[2] u[2]
  \\
  \\
  x[4] &= A[3] x[3] + B[3] u[3] = A[3] A[2] A[1] B[0] u[0] + A[3] A[2] B[1] u[1] + A[3] B[2] u[2] + B[3] u[3]
  \\
  \\
  &= \Phi[4,1] B[0] u[0] + \Phi[4,2]  B[1] u[1] + \Phi[4,3] B[2] u[2] + \Phi[4,4] B[3] u[3]
  \\
  \vdots
 \\
  x[k] &= \Phi[k,1] B[0] u[0] + \Phi[k,2] B[1] u[1] + \cdots + \Phi[k,k-1] B[k-2] u[k-2] + \Phi[k,k] B[k-1] u[k-1]
         \\
         &= \left[ \begin{array}{c|c|c|c|c} \Phi[k,1] B[0] & \Phi[k,2] B[1]  & \cdots & \Phi[k,k-1] B[k-2]  & \Phi[k,k] B[k-1] \end{array} \right]
         \left[ \begin{array}{c}
                  u[0] \\ u[1] \\ \vdots \\ u[k-2] \\ u[k-1]
         \end{array} \right]
         \\
         &= \sum\limits_{j = 0}^{k-1} \Phi[k,j+1] B[j] u[j]
         \\
         &= \Gamma[k,0] \mathcal{U}[k,0]
\end{align*}
%
where
%
\begin{align*}
    \Gamma[k,0] &= \left[ \begin{array}{c|c|c|c|c} \Phi[k,1] B[0] & \Phi[k,2] B[1]  & \cdots & \Phi[k,k-1] B[k-2]  & \Phi[k,k] B[k-1] \end{array} \right]
\\
    \mathcal{U}[k,0] &=  \left[ \begin{array}{c} u[0] \\ u[1] \\ \vdots \\ u[k-2] \\ u[k-1]  \end{array} \right]
\end{align*}
%

\begin{exmp}
Let's consider following SISO LTP system
 \begin{align*}
  x[k+1] &= A[k] x[k] + B[k] u[k] , \ , y[k] = C[k] x[k]  \ , \   \mathrm{where} \\
  A[k] &= A[k+4]  \ , \  B[k] = B[k+4]  \ , \ C[k] = C[k+4]
 \end{align*}
\end{exmp}
%
convert this SISO LTP system into a MISO LTI system

\textbf{Solution:} Let's derive $x[4]$ in terms of $x[0] , u[0] , u[1], u[2] , u[3]$

\begin{align*}
 x[4] &= A[3] A[2] A[1] A[0] x[0] +  A[3] A[2] A[1] B[0] u[0] + A[3] A[2] B[1] u[1] + A[3] B[2] u[2] + B[3] u[3]
 \\
 \\
  &= \begin{bmatrix} A[3] A[2] A[1] A[0] \end{bmatrix} x[0] + \begin{bmatrix} A[3] A[2] A[1] B[0] & A[3] A[2] B[1] & A[3] B[2] &  B[3] \end{bmatrix} 
 \begin{bmatrix} u[0] \\ u[1] \\ u[2] \\ u[3] \end{bmatrix} 
\end{align*}
%
Now let's find $x[8]$ in terms of $x[4] , u[4] , u[5], u[6] , u[7]$
%
\begin{align*}
 x[8] &= \begin{bmatrix} A[3] A[2] A[1] A[0] \end{bmatrix} x[4] + \begin{bmatrix} A[3] A[2] A[1] B[0] & A[3] A[2] B[1] & A[3] B[2] &  B[3] \end{bmatrix} 
 \begin{bmatrix} u[4] \\ u[5] \\ u[6] \\ u[7] \end{bmatrix} 
\end{align*}
%
Let $\mathcal{X}[m] = x[4k]$,  $\mathcal{U}[m] = \begin{bmatrix} u[4k] & u[4k + 1] & u[4k + 2] & u[4k + 3] \end{bmatrix}^T$, and $\mathcal{Y}[m] = y[4k]$, then we have 
%
\begin{align*}
 \mathcal{X}[m+1] &= \begin{bmatrix} A[3] A[2] A[1] A[0] \end{bmatrix} \mathcal{X}[m] + \begin{bmatrix} A[3] A[2] A[1] B[0] & A[3] A[2] B[1] & A[3] B[2] &  B[3] \end{bmatrix} 
\mathcal{U}[m] 
\\
 &= \mathcal{A}  \mathcal{X}[m] + \mathcal{B} \mathcal{U}[m] 
\\ 
\mathcal{Y}[m] &= \left[ C[0] \right] \mathcal{X}[m]
\\
&= \mathcal{C} \mathcal{X}[m]
\end{align*}
%
which is a multi input single output LTI state-space representation. Note that in this representation we technically perform periodic sampling and analyze the mapping between two sampling instants. However, in this representation we technically loose some information, e.g. measurements/outputs between to sampling instants at i.e. $y[4k+1] \ , \ y[4k+2] \, \ y[4k+3] \ k > 0$.

\begin{exmp}
	Obtain a new state-space representation (still LTI) such that it also includes these measurements, i.e. $y[4k+1] \ , \ y[4k+2] \, \ y[4k+3] \ k > 0$.
\end{exmp}

% **** This ENDS THE EXAMPLES. DON'T DELETE THE FOLLOWING LINE:
\end{document}
