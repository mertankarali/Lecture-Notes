

\documentclass[11pt,oneside]{amsart}
\usepackage[margin=1in]{geometry}
\usepackage{times}
\usepackage{graphicx}
\usepackage{amsmath,amsfonts}
\usepackage{url}
\usepackage[colorlinks=true,
            linkcolor=red,
            urlcolor=blue,
            citecolor=gray]{hyperref}
\parindent=0pt
\pagestyle{empty}

\newcommand{\header}[1]{\bigbreak\textbf{#1}}

\begin{document}

\begin{center}
  \bf
  \includegraphics[width=3in]{metu_logo} \\
  EE502 Syllabus \\
 Electrical and Electronics Engineering \\ 
Linear Systems Theory II - 5670502 \\
Spring, 2024-2025
\end{center}

\header{Instructor} 

\vspace{4pt}

Assoc. Prof. M. Mert Ankarali

e-mail: \url{mertan@metu.edu.tr}.

\header{Course Details}

\vspace{6pt}

\textbf{Prerequisite:} Since it is a graduate-level course, there is no officially defined prerequisite course. However, it is highly recommended that students are either comfortable with or at least familiar with the contents covered in the following METU-EEE courses.

\vspace{6pt}

\begin{itemize}
  \item \textbf{EE302 - Feedback Systems.} Fundamental level undergraduate control systems course that teaches the basics of modeling, analysis, and design of (Continuous Time - LTI) control systems. Note that it provides a significantly more in-depth and comprehensive treatment of the subject than other undergraduate control systems courses, both within this university and likely at other institutions.
   \item \textbf{EE402 - Discrete-Time Systems.} An upper-level undergraduate course on discrete-time control systems. Students learn the fundamentals of modeling, analysis, and design of digital and discrete-time control systems. Note: This course is one of the most, if not \textit{the} most, rigorous undergraduate offerings in control theory and dynamical systems at this university, both in breadth and depth.
   \item \textbf{EE501 - Linear Systems Theory.} A graduate-level course dedicated to the theoretical development of linear algebra, covering fundamental concepts with limited emphasis on practical applications.
\end{itemize}

\vspace{9pt}

\textbf{Lecture Hours:}, Tuesday 14:40-16:30, Thursday 8:40-10:30 @EEMB D-134

\vspace{9pt}

\textbf{Lecture Notes:} Lecture notes for this course are available on a public GitHub repository. These notes, originally developed for a previous offering of EE502, are a work in progress and may contain errors.  The GitHub repository will be updated throughout the semester with new material and corrections.  Students are invited to contribute to the ongoing improvement of these notes by submitting corrections and suggestions via the repository.

\vspace{6pt}

\textbf{GitHub Repository Link:} \url{https://github.com/mertankarali/Lecture-Notes/}

\vspace{6pt}

\textbf{Textbook:} J. P. Hespanha. Linear systems theory. Princeton
University Press, 2009.

\vspace{6pt}

\textbf{Auxiliary Sources:}

\vspace{6pt}

[A1] 6.241- Dynamic systems and control (MIT OpenCourseWare),
    M. Dahleh, M.A. Dahleh and G. C. Verghese. 

\url{ocw.mit.edu/courses/6-241j-dynamic-systems-and-control-spring-2011/}

\vspace{6pt}

[A2] E. Tuna, EE502 Course Page. \url{users.metu.edu.tr/home202/etuna/wwwhome/ee502/}

\newpage

\header{Description \& Outline}

\vspace{6pt}

EE502 is a graduate-level course in control and dynamical systems theory. This course provides a rigorous introduction to the theoretical foundations of dynamical systems, with a focus on linear time-invariant (LTI) systems.  Key topics include system representations, stability, reachability \& observability, state feedback, state estimation, and realization.

\header{Course Grading}

\begin{itemize}

\item \textbf{2 Midterm Examinations ($M_1 \& M_2$) and 1 Final Exam ($F$):} There
  will be 2 on-campus midterm examinations. 

\vspace{3pt}


\begin{itemize}

    \item Each exam will primarily utilize a \textbf{limited-open} material exam format. However, some exams may include closed-format questions.

    \item In a \textbf{limited-open} material exam, you may use any text-based material (notes, books, worksheets, etc.) and any computing device without internet or wireless communication capabilities (e.g., calculators).

    \item \textbf{Closed-format} questions prohibit all materials, including books, notes, and calculators.

    \item In the past, I have used \textbf{general-open} and take-home exams in EE502 and similar courses. I may use these formats for one or two midterm exams.

\end{itemize}

\vspace{6pt}

\item \textbf{Total Course Grade ($G$)} is calculated based on the
  formula given below
%
\begin{align*}
  G = 0.3 \, \mathbf{M_1} + 0.3 \, \mathbf{M_2} + 0.4 \, \mathbf{F} 
\end{align*}

\end{itemize}

\end{document}
\end{document}
