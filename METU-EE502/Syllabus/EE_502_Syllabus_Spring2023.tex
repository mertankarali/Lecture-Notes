\documentclass[11pt,oneside]{amsart}
\usepackage[margin=1in]{geometry}
\usepackage{times}
\usepackage{graphicx}
\usepackage{amsmath,amsfonts}
\usepackage{url}
\usepackage[colorlinks=true,
            linkcolor=red,
            urlcolor=blue,
            citecolor=gray]{hyperref}
\parindent=0pt
\pagestyle{empty}

\newcommand{\header}[1]{\bigbreak\textbf{#1}}

\begin{document}

\begin{center}
  \bf
  \includegraphics[width=3in]{metu_logo} \\
  Syllabus \\
 Electrical and Electronics Engineering \\ 
Linear Systems Theory II - 5670502 \\
Spring, 2022
\end{center}

\header{Instructor} 

\vspace{4pt}

Assoc. Prof. M. Mert Ankarali

e-mail: \url{mertan@metu.edu.tr}.

\header{Course Details}

\vspace{6pt}

Until further notice from the university, I will teach the content
synchronously from the officially defined classroom and stream the
lectures via Zoom video conference software (I will announce Zoom
links via the ODTUclss page of the course). Note that attendance will
\textbf{never} be mandatory throughout the course (attendance has
never been compulsory in classes I offer alone, anyway). However,  I
highly recommend regularly following lectures, and I believe it is
pointless to miss graduate-level lecture sessions regularly. 

\vspace{3pt}

Note that, attendance to physical on-site midterms and final exam will
be obviously mandatory unless you have official medical reasoning. 

\vspace{6pt}

\textbf{Prerequisite:} Since it is a graduate-level course, there is no officially defined prerequisite course. However, it is highly recommended that students are either comfortable with or at least familiar with the contents covered in the following METU-EEE courses


\begin{itemize}
  \item \textbf{EE302 - Feddback Systems.} Fundemental level undergraduate control
    systems course that teach the basics of modeling, analysis
    and design of (Continious Time - LTI) control systems. 
   \item \textbf{EE402 - Disctrete-Time Systems.} Upper level undergraduate control
    systems course that teach the fundementals of modeling, analysis
    and design of \textbf{digital \& discrete-time} control systems. 
   \item \textbf{EE501 - Linear Systems Theary.} Graduate 
    level linear-algebra course.
\end{itemize}

\vspace{9pt}

\textbf{Lecture Hours:}, Tuesday 13:40-15:30, Thursday 11:40-12:30
@EEMB A-306

\vspace{9pt}

\textbf{Lecture Notes:} In my courses, I generally create original
lecture notes and post them on a public GitHub page. I am offering
this course for the first time, and unfortunately, the start of the
semester was very erratic. I will still try to create notes for each
lecture and publish them on the GitHub page. However, it will be
possible that I won't be able to post the notes of some lectures, or
I may share some lectures after the synchronous classes. Moreover, always follow the changes in the GitHub repository. 

\vspace{6pt}

\textbf{GitHub Repository Link:} \url{https://github.com/mertankarali/Lecture-Notes/}

\vspace{6pt}

\textbf{Textbook:} J. P. Hespanha. Linear systems theory. Princeton
University Press, 2009.

\vspace{6pt}

\textbf{Auxiliary Sources:}

\begin{itemize}
  \item 6.241- Dynamic systems and control (MIT Opencourseware),
    M. Dahleh, M.A. Dahleh and G. C. Verghese. \url{https://ocw.mit.edu/courses/6-241j-dynamic-systems-and-control-spring-2011/}
   \item E. Tuna, EE502 Course Page. \url{https://users.metu.edu.tr/home202/etuna/wwwhome/ee502/}
\end{itemize}

\newpage

\header{Description \& Outline}

\vspace{6pt}

EE502 is a graduate-level course in the area of control and dynamical
systems theory. 
The main goal is to teach the fundamental concepts in dynamical systems (with
main focus on LTI systems) such as system representations, stability,
reachability \& observability, state-feedback, state-estimation,
realization. 

\vspace{6pt}

\header{Course Grading}

\begin{itemize}

\item \textbf{2 Midterm Examinations ($M_1 \& M_2$) and 1 Final Exam ($F$):} There
  will be 2 on-campus midterm examinations. 

\vspace{3pt}

\begin{itemize}

\item Each exam will mainly rely on \textbf{limited-open} material
  exam concept. However in some of the (or all of the) exams, there
  could be closed-format questions.

\item In a \textbf{limited-open} material exam you can
use any text-based material (notes, book, worksheets, etc.) and any computing
device provided it does not have internet and/or wireless communication capabilities 
(e.g., calculators).

\item In \textbf{closed-format} questions, all materials (including
 books, notes, calculators) will be prohibited. 

\end{itemize}

\vspace{6pt}

\item \textbf{Total Course Grade ($G$)} is calculated based on the
  formula given below
%
\begin{align*}
  G = 0.3 \, \mathbf{M_1} + 0.3 \, \mathbf{M_2} + 0.40 \, \mathbf{F} 
\end{align*}

\end{itemize}

\end{document}
