

\documentclass[11pt,oneside]{amsart}
\usepackage[margin=1in]{geometry}
\usepackage{times}
\usepackage{graphicx}
\usepackage{amsmath,amsfonts}
\usepackage{url}
\usepackage[colorlinks=true,
            linkcolor=red,
            urlcolor=blue,
            citecolor=gray]{hyperref}
\parindent=0pt
\pagestyle{empty}

\newcommand{\header}[1]{\bigbreak\textbf{#1}}

\begin{document}

\begin{center}
  \bf
  \includegraphics[width=3in]{metu_logo} \\
  EE502 Syllabus \\
 Electrical and Electronics Engineering \\ 
Linear Systems Theory II - 5670502 \\
Spring, 2023
\end{center}

\header{Instructor} 

\vspace{4pt}

Assoc. Prof. M. Mert Ankarali

e-mail: \url{mertan@metu.edu.tr}.

\header{Course Details}

\vspace{6pt}

\textbf{Prerequisite:} Since it is a graduate-level course, there is no officially defined prerequisite course. However, it is highly recommended that students are either comfortable with or at least familiar with the contents covered in the following METU-EEE courses.


\begin{itemize}
  \item \textbf{EE302 - Feedback Systems.} Fundamental level undergraduate control
    systems course that teaches the basics of modeling, analysis
    , and design of (Continuous Time - LTI) control systems. 
   \item \textbf{EE402 - Discrete-Time Systems.} Upper-level undergraduate control
    systems course that teaches the fundamentals of modeling, analysis
    , and design of \textbf{digital \& discrete-time} control systems. 
   \item \textbf{EE501 - Linear Systems Theory.} Graduate-level linear algebra course.
\end{itemize}

\vspace{9pt}

\textbf{Lecture Hours:}, Tuesday 9:40-12:30, Thursday 13:40-15:30
@EEMB A-306

\vspace{9pt}

\textbf{Lecture Notes:} I create original lecture notes in my courses and post them on a public GitHub page. Last year (when I offered EE502 for the first time), I created lecture notes, and they are currently available in the GitHub repository. However, since I am offering the course only a second time, there could be many typos/mistakes. Moreover, I can make some new changes and additions to the notes throughout the semester. In that respect, you should follow the updates in the GitHub repository. 


\vspace{6pt}

\textbf{GitHub Repository Link:} \url{https://github.com/mertankarali/Lecture-Notes/}

\vspace{6pt}

\textbf{Textbook:} J. P. Hespanha. Linear systems theory. Princeton
University Press, 2009.

\vspace{6pt}

\textbf{Auxiliary Sources:}

\vspace{6pt}

[A1] 6.241- Dynamic systems and control (MIT OpenCourseWare),
    M. Dahleh, M.A. Dahleh and G. C. Verghese. 

\url{ocw.mit.edu/courses/6-241j-dynamic-systems-and-control-spring-2011/}

\vspace{5pt}

[A2] E. Tuna, EE502 Course Page. \url{users.metu.edu.tr/home202/etuna/wwwhome/ee502/}


\header{Description \& Outline}

\vspace{6pt}

EE502 is a graduate-level course in the area of control and dynamical
systems theory. 
The main goal is to teach the fundamental concepts in dynamical systems (with
main focus on LTI systems) such as system representations, stability,
reachability \& observability, state feedback, state estimation,
realization. 

\newpage

\header{Course Grading}

\begin{itemize}

\item \textbf{2 Midterm Examinations ($M_1 \& M_2$) and 1 Final Exam ($F$):} There
  will be 2 on-campus midterm examinations. 

\vspace{3pt}

\begin{itemize}

\item Each exam will mainly rely on \textbf{limited-open} material
  exam concept. However, there could be closed-format questions in some of the (or all of the) exams.

\item In a \textbf{limited-open} material exam you can
use any text-based material (notes, books, worksheets, etc.) and any computing
device provided it does not have internet and/or wireless communication capabilities 
(e.g., calculators).

\item In \textbf{closed-format} questions, all materials (including
 books, notes, calculators) will be prohibited. 

\end{itemize}

\vspace{6pt}

\item \textbf{Total Course Grade ($G$)} is calculated based on the
  formula given below
%
\begin{align*}
  G = 0.3 \, \mathbf{M_1} + 0.3 \, \mathbf{M_2} + 0.4 \, \mathbf{F} 
\end{align*}

\end{itemize}

\end{document}
\end{document}
