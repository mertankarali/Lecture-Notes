%
% This is a borrowed LaTeX template file for lecture notes for CS267,
% Applications of Parallel Computing, UCBerkeley EECS Department.
% Now being used for CMU's 10725 Fall 2012 Optimization course
% taught by Geoff Gordon and Ryan Tibshirani.  When preparing 
% LaTeX notes for this class, please use this template.
%
% To familiarize yourself with this template, the body contains
% some examples of its use.  Look them over.  Then you can
% run LaTeX on this file.  After you have LaTeXed this file then
% you can look over the result either by printing it out with
% dvips or using xdvi. "pdflatex template.tex" should also work.
%

\documentclass[twoside]{article}
\setlength{\oddsidemargin}{0.25 in}
\setlength{\evensidemargin}{-0.25 in}
\setlength{\topmargin}{-0.6 in}
\setlength{\textwidth}{6.5 in}
\setlength{\textheight}{8.5 in}
\setlength{\headsep}{0.75 in}
\setlength{\parindent}{0 in}
\setlength{\parskip}{0.1 in}

%
% ADD PACKAGES here:
%

\usepackage{amsmath,amsfonts,graphicx}

%
% The following commands set up the lecnum (lecture number)
% counter and make various numbering schemes work relative
% to the lecture number.
%
\newcounter{lecnum}
\renewcommand{\thepage}{\thelecnum-\arabic{page}}
\renewcommand{\thesection}{\thelecnum.\arabic{section}}
\renewcommand{\theequation}{\thelecnum.\arabic{equation}}
\renewcommand{\thefigure}{\thelecnum.\arabic{figure}}
\renewcommand{\thetable}{\thelecnum.\arabic{table}}

%
% The following macro is used to generate the header.
%
\newcommand{\lecture}[4]{
   \pagestyle{myheadings}
   \thispagestyle{plain}
   \newpage
   \setcounter{lecnum}{#1}
   \setcounter{page}{1}
   \noindent
   \begin{center}
   \framebox{
      \vbox{\vspace{2mm}
    \hbox to 6.28in { {\bf EE402 - Discrete Time Systems
	\hfill Spring 2018} }
       \vspace{4mm}
       \hbox to 6.28in { {\Large \hfill Lecture #1 \hfill} }
       \vspace{2mm}
       \hbox to 6.28in { {\it Lecturer: #2 \hfill } }
      \vspace{2mm}}
   }
   \end{center}
   \markboth{Lecture #1}{Lecture #1}

   \vspace*{4mm}
}
%
% Convention for citations is authors' initials followed by the year.
% For example, to cite a paper by Leighton and Maggs you would type
% \cite{LM89}, and to cite a paper by Strassen you would type \cite{S69}.
% (To avoid bibliography problems, for now we redefine the \cite command.)
% Also commands that create a suitable format for the reference list.
\renewcommand{\cite}[1]{[#1]}
\def\beginrefs{\begin{list}%
        {[\arabic{equation}]}{\usecounter{equation}
         \setlength{\leftmargin}{2.0truecm}\setlength{\labelsep}{0.4truecm}%
         \setlength{\labelwidth}{1.6truecm}}}
\def\endrefs{\end{list}}
\def\bibentry#1{\item[\hbox{[#1]}]}

%Use this command for a figure; it puts a figure in wherever you want it.
%usage: \fig{NUMBER}{SPACE-IN-INCHES}{CAPTION}
\newcommand{\fig}[3]{
			\vspace{#2}
			\begin{center}
			Figure \thelecnum.#1:~#3
			\end{center}
	}
% Use these for theorems, lemmas, proofs, etc.
\newtheorem{theorem}{Theorem}[lecnum]
\newtheorem{lemma}[theorem]{Lemma}
\newtheorem{proposition}[theorem]{Proposition}
\newtheorem{claim}[theorem]{Claim}
\newtheorem{corollary}[theorem]{Corollary}
\newtheorem{definition}[theorem]{Definition}
\newenvironment{proof}{{\bf Proof:}}{\hfill\rule{2mm}{2mm}}

% **** IF YOU WANT TO DEFINE ADDITIONAL MACROS FOR YOURSELF, PUT THEM HERE:

\begin{document}

% Lecture Details
\lecture{8}{Asst. Prof. M. Mert Ankarali}

\par

\section*{Stability of Discrete Time Control Systems}

For an LTI discrete time dynamical system which can be represented
with a rational transfer function, closed loop poles determine the
stability characteristics of the system.
%
\begin{itemize}
  \item If all poles of the system are located strictly inside the
    unit-circle then the system is \textbf{(asymptotically) stable}.
   Asymptotically stability systems are also \textbf{BIBO stable}. 
  \item If there exist some \textit{simple} (non-repeated) poles on 
   the unit circle and all remaining poles are located inside the
   unit circle, then the system is \textbf{critically/marginally
     stable}. Note that critically/marginally stable systems are 
    \textbf{BIBO unstable}.
  \item If there exist at least one repeated pole on the 
   the unit circle, then the system is \textbf{unstable}, of course
   also \textbf{BIBO unstable}.
   \item If there exist at least one pole outside of the unit circle, 
      then the system is \textbf{unstable}, of course
   also \textbf{BIBO unstable}.
\end{itemize}

\subsection*{Jury Stability Test}

Jury stability test similar to the Routh-Hurwitz in CT systems,
can define the stability of a DT system given the characteristic
equation which is in the form 
%
\begin{align*}
  D(z) = a_0 z^n + a_1 z^{n-1} + \cdots + a_{n-1} z + a_n
\end{align*}
%
without loss of generality we will assume that $a_0 > 0$.

\paragraph{First Order:} When $n=1$, $D(z)$ takes the form
%
\begin{align*}
  D(z) = a_0 z + a_1
\end{align*}
%
DT System is stable if
%
\begin{align*}
  |a_1| < a_0 
\end{align*}
%
\paragraph{Second Order:} When $n=2$, $D(z)$ takes the form
%
\begin{align*}
  D(z) = a_0 z^2 + a_1 z + a_2
\end{align*}
%
DT System is stable if
%
\begin{align*}
  |a_2| &< a_0 
\\
D(1) &> 0
\\
D(-1) &> 0
\end{align*}
%
\paragraph{Third Order:} When $n=3$, $D(z)$ takes the form
%
\begin{align*}
  D(z) = a_0 z^3 + a_1 z^2 + a_2 z + a_3
\end{align*}
%
We need to construct the Jury table 
%
\begin{center}
  \begin{tabular}{ | c || c  c  c c |}
    \hline
    Row & $z^0$ & $z^1$ & $z^2$ & $z^3$ \\ \hline \hline
    1 & $a_3$ & $a_2$ & $a_1$ & $a_0$ \\ \hline
    2 & $a_0$ & $a_1$ & $a_2$ & $a_3$ \\ \hline
    3 & $b_2$ & $b_1$ & $b_0$ &  \\\hline
  \end{tabular}
\end{center}
%
where
%
\begin{align*}
  b_{0} = \left| \begin{array}{cc} a_3 & a_2 \\ a_0 & a_1 \end{array}
                                                      \right|
\quad , \quad 
  b_{1} = \left| \begin{array}{cc} a_3 & a_1 \\ a_0 & a_2 \end{array}
                                                      \right|
\quad , \quad 
  b_{2} = \left| \begin{array}{cc} a_3 & a_0 \\ a_0 & a_3 \end{array}
                                                      \right|
\end{align*}
%
Then DT system is stable if
%
\begin{align*}
  \mathbf{|a_3|} &< a_0 
\\
D(1) &> 0
\\
\pmb{-} D(-1) &> 0
\\
|b_2| &> |b_0|
\end{align*}
%

\paragraph{General Case:} 
%
The jury table for systems with order $n$ has $2n - 3$
rows and it has the form below
%
\begin{center}
  \begin{tabular}{ | c || c c c c c c c|}
    \hline
    Row & $z^0$ & $z^1$ & $z^2$ & $\cdots$ & $z^{n-2}$ & $z^{n-1}$ & $z^n$ \\ \hline \hline
    1 & $a_n$ & $a_{n-1}$ & $a_{n-2}$ & $\cdots$ & $a_2$ & $a_1$ & $a_0$ \\ \hline
    2 & $a_0$ & $a_1$ & $a_2$ & $\cdots$ & $a_{n-2}$ & $a_{n-1}$ & $a_{n}$ \\ \hline
    3 & $b_{n-1}$ & $b_{n-2}$ & $b_{n-3}$ & $\cdots$ & $b_1$ & $b_0$ &
                                                                        \\ \hline
    4 & $b_0$ & $b_1$ & $b_2$ & $\cdots$ & $b_{n-2}$ & $b_{n-1}$ & \\
    \hline
    5 & $c_{n-2}$ & $c_{n-3}$ & $c_{n-3}$ & $\cdots$ & $c_0$ &  &  \\ \hline
    6 & $c_0$ & $c_1$ & $c_2$ & $\cdots$ & $c_{n-2}$ &  & \\ \hline
$\vdots$ & $\vdots$ & & & & &  & \\ \hline
    2n - 3 & $q_2$ & $q_1$ & $q_0$ & & & &  \\ \hline
  \end{tabular}
\end{center}
%
where
%
\begin{align*}
  b_{k} &= \left| \begin{array}{cc} a_n & a_{n-1-k} \\ a_0 &
       a_{k+1} \end{array} \right| \quad , \quad  k \in \lbrace 0,1,
                                                            \cdots,
                                                            n-1
                                                            \rbrace
\\
  c_{k} &= \left| \begin{array}{cc} b_{n-1} & b_{n-2-k} \\ b_0 &
       b_{k+1} \end{array} \right| \quad , \quad  k \in \lbrace 0,1,
                                                            \cdots,
                                                            n-2
                                                            \rbrace
\\
  q_{k} &= \left| \begin{array}{cc} p_3 & p_{2-k} \\ p_0 &
       p_{k+1} \end{array} \right| \quad , \quad  k \in \lbrace 0,1,3 \rbrace
\end{align*}
%
Then DT system is stable if
%
\begin{align*}
|a_n| &< a_0 
\\
D(1) &> 0
\\
(-1)^n D(-1) &> 0
\\
|b_{n-1}| &> |b_0|
\\
|c_{n-2}| &> |c_0|
\\
\cdots
\\
|q_{2}| &> |q_0|
\end{align*}
%

\textbf{Example:} Using Jury test, find if the following 
characteristic equation is stable or not
%
\begin{align*}
  G(z) = \frac{0.02 z^{-1} + 0.03 z^{-2} + 0.02 z^{-3}}{1 - 3 z^{-1} + 4 z^{-2} - 2 z^{-3} + 0.5 z^{-4}}
\end{align*}
%

\textbf{Solution:} This is a $4^{th}$ order system for which the
characteristic equation is 
%
\begin{align*}
  D(z) &= a_0 z^4 + a_1 z^3 + a_2 z^2 + a_3 z +  a_4\\
         &= 1 z^4 + -3 z^3 + 4 z^2 + -2 z + 0.5
\end{align*}
%
Jury table for a $n=4$ system has the form
%
\begin{center}
  \begin{tabular}{ | c || c c c c c |}
    \hline
    Row & $z^0$ & $z^1$ & $z^2$ & $z^3 $ & $z^4$ \\ \hline \hline
    1 & $a_4$ & $a_3$ & $a_2$ & $a_1$ & $a_0$ \\ \hline
    2 & $a_0$ & $a_1$ & $a_2$ & $a_3$ & $a_4$ \\ \hline
    3 & $b_3$ & $b_2$ & $b_1$ & $b_0$ &  \\ \hline
    4 & $b_0$ & $b_1$ & $b_2$ & $b_3$ &  \\ \hline
    5 & $c_2$ & $c_1$ & $c_0$ &  &  \\ \hline
  \end{tabular}
\end{center}
%
Before computing the whole Jury table let's check conditions
one-by-one
%
\begin{itemize}
 \item Check if $|a_4| < a_0 $
%
\begin{align*}
0.5 < 1 \quad \mathrm{OK}
\end{align*}
%
\item Check if $D(1) > 0$
%
\begin{align*}
  D(1) = 1 - 3 + 4 -2 + 0.5 = 0.5 > 0 \quad \mathrm{OK}
\end{align*}
%
\item Check if $(-1)^4 D(-1) > 0$
%
\begin{align*}
  D(-1) = 1 + 3 + 4 + 2 + 0.5 = 10.5 > 0 \quad \mathrm{OK}
\end{align*}
%
\item Let's compute $b_0$ and $b_3$ and check if $|b_3| > |b_0| $
%
\begin{align*}
  b_0 &= \left| \begin{array}{cc} a_4 & a_3 \\ a_0 & a_1 \end{array}
                                                     \right| = 
\left| \begin{array}{cc} 0.5 & -2 \\ 1 & -3 \end{array}
                                                     \right| = 0.5
\\
b_3 &= \left| \begin{array}{cc} a_4 & a_0 \\ a_0 & a_4 \end{array}
                                                     \right| =  \left| \begin{array}{cc} 0.5 & 1 \\ 1 & 0.5 \end{array}
                                                     \right| = -0.75
\\
|b_3| &= 0.75 > 0.5 =  |b_0| \quad \mathrm{OK}
\end{align*}
%
\item Let's compute $b_1$ and $b_2$ 
%
\begin{align*}
  b_1 &= \left| \begin{array}{cc} a_4 & a_2 \\ a_0 & a_2 \end{array}
                                                     \right| = 
\left| \begin{array}{cc} 0.5 & 4 \\ 1 & 4 \end{array}
                                                     \right| = -2
\\
b_2 &= \left| \begin{array}{cc} a_4 & a_1 \\ a_0 & a_3 \end{array}
                                                     \right| =  \left| \begin{array}{cc} 0.5 & -3 \\ 1 & -2 \end{array}
                                                     \right| = 2
\end{align*}
%
%
\item Let's compute $c_0$ and $c_2$ and check if $|c_2| > |c_0| $
%
\begin{align*}
  c_{0} &= \left| \begin{array}{cc} b_3 & b_2 \\ b_0 & b_1 \end{array}
                                                      \right| =
\left| \begin{array}{cc} -0.75 & 2 \\ 0.5 & -2 \end{array}
                                                      \right| = 0.5
\\
  c_{2} &= \left| \begin{array}{cc} b_3 & b_0 \\ b_0 & b_3 \end{array}
                                                      \right|
= \left| \begin{array}{cc} -0.75 & 0.5 \\ 0.5 & -0.75 \end{array}
                                                      \right| = 0.3125
\\
|c_2| &= 0.3125 \not> 0.5 =  |c_0| \quad \mathrm{NOT \ OK}
\end{align*}
%

Final Jury Table is also given below
%
\begin{center}
  \begin{tabular}{ | c || c c c c c |}
    \hline
    Row & $z^0$ & $z^1$ & $z^2$ & $z^3 $ & $z^4$ \\ \hline \hline
    1 & $a_4 = 0.5$ & $a_3 = -2$ & $a_2 = 4$ & $a_1 = -3$ & $a_0 = 1$ \\ \hline
    2 & $a_0 = 1$ & $a_1 = -3$ & $a_2 = 4$ & $a_3 = -2$ & $a_4 = 0.5$ \\ \hline
    3 & $b_3 = -0.75$ & $b_2 = 2$ & $b_1 = -2$ & $b_0 = 0.5$ &  \\ \hline
    4 & $b_0 = 0.5$ & $b_1 = -2$ & $b_2 = 2$ & $b_3 = -0.75$ &  \\ \hline
    5 & $c_2 = 0.3125$ & $c_1$ & $c_0 = 0.5$ &  &  \\ \hline
  \end{tabular}
\end{center}
%

\end{itemize}
%


\subsection*{Bilinear Transformation \& Routh-Hurwitz Test}

In Lecture 7, we showed that bilinear transformation has a 1-1 
mapping between stable regions in z-plane and s-plane, as well as
unstable regions in z-plane and s-plane. As a way of testing
stability, we can transform the characterstic polynomial 
using bilinear transformation, then we can apply Routh-Hurwitz
test. 

Routh-Hurwitz is simpler and easier then the Jury test,
however amount of computation needed for transformation
generally shadows the relative computational advantage of 
Routh-Hurwitz.

We know that the bilinear transformation has the form
%
\begin{align*}
 z &= \frac{1+\frac{T}{2} \bar{s}}{1-\frac{T}{2} \bar{s}}
\end{align*}
%
Since we only consider the test of stability, for the sake of
simplicity it is reasonable to assume that $T = 2$. Then, the
transformation of a general $D(z)$ looks like
%
%
\begin{align*}
D(\bar{s}) = D(z)|_{z = \frac{1+\bar{s}}{1-\bar{s}}} = 
a_0 \left( \frac{1+\bar{s}}{1-\bar{s}} \right)^n + a_1 \left(
  \frac{1+\bar{s}}{1-\bar{s}} \right)^{n-1} 
+ \cdots + a_{n-1} \left( \frac{1+\bar{s}}{1-\bar{s}} \right) + a_n
\end{align*}
% 
Then clearing the fractions by multiplying both sides by $(1 -
\bar{s})^n$, we obtain
%
\begin{align*}
Q(\bar{s}) = b_0 \bar{s}^n + b_1 \bar{s}^{n-1} + \cdots + b_{n-1}
  \bar{s} + b_n
\end{align*}
%
Testing the stability on $Q(s)$ using Routh-Hurwitz will yield
the stability condition of the original DT system. 

\textbf{Example:} Consider the following characterstic equation 
of a DT system
%
\begin{align}
  D(z) = (z-1)*(z-2) = z^2 -3z +2
\end{align}
%
Test the stability (already known) using Bilinear Transformation and 
Routh-Hurwitz.

\textbf{Solution:} 
%
\begin{align*}
D(\bar{s}) &= D(z)|_{z = \frac{1+\bar{s}}{1-\bar{s}}} =
\left( \frac{1+\bar{s}}{1-\bar{s}} \right)^2 - 3 \left(
  \frac{1+\bar{s}}{1-\bar{s}} \right) + 2
\\
Q(\bar{s}) &= \left( 1+\bar{s} \right)^2 - 3 (1+\bar{s} )( 1-\bar{s}) +
  2 \left( 1 - \bar{s} \right)^2
\\
&= (1 + 2 \bar{s} + \bar{s}^2) - 3 (1 - \bar{s}^2 )  + 2 ( 1 - 2
  \bar{s} + \bar{s}^2 )
\\
&= 6 \bar{s}^2 - 2 \bar{s} 
\end{align*}
%
This artificial CT system is unstable since one coefficient
is negative and one coefficient is equal to zero. It is clear
from this example that just for testing stability Bilinear 
transformation is not very useful. 

% **** This ENDS THE EXAMPLES. DON'T DELETE THE FOLLOWING LINE:
\end{document}
